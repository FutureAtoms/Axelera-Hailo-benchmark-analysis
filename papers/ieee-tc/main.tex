% IEEE Transactions on Computers LaTeX Template
% Corrected version addressing all review issues
\documentclass[journal]{IEEEtran}

% Required packages
\usepackage{cite}
\usepackage{amsmath,amssymb,amsfonts}
\usepackage{algorithmic}
\usepackage{graphicx}
\usepackage{textcomp}
\usepackage{xcolor}
\usepackage{booktabs}
\usepackage{multirow}
\usepackage{url}
\usepackage{hyperref}

% Math environments
\usepackage{amsthm}
\newtheorem{theorem}{Theorem}
\newtheorem{lemma}{Lemma}
\newtheorem{definition}{Definition}

\begin{document}

\title{Comprehensive Performance Evaluation of Axelera AI Metis Edge AI Accelerator with Statistical Validation}

\author{
    \IEEEauthorblockN{Abhilash Chadhar}
    \IEEEauthorblockA{
        Axelera AI\\
        High Tech Campus 5, 5656 AE Eindhoven, The Netherlands\\
        Email: abhilashchadhar@gmail.com\\
        ORCID: 0009-0003-7656-8161
    }
}

\maketitle

\begin{abstract}
We present a comprehensive performance evaluation of the Axelera AI Metis edge AI accelerator through systematic hardware testing with 1,199 real measurements. Our empirical evaluation reveals peak throughput of 6,829.2 FPS on ResNet-18, power efficiency of 228.26 FPS/W, and 79.9\% multi-core scaling efficiency across four cores. Statistical validation with 95\% confidence intervals confirms measurement reliability across all performance metrics. The evaluation methodology employs substantially larger sample sizes than typical practice (1,199 vs. common 5-50 measurements) to enable robust statistical analysis. We provide detailed performance characterization across multiple neural network models, core configurations, and batch sizes under controlled thermal conditions. The complete dataset, analysis scripts, and reproducible methodology are made available to support independent validation and comparative studies. This work contributes empirical performance data essential for informed hardware selection in edge AI applications.
\end{abstract}

\begin{IEEEkeywords}
AI accelerators, edge computing, performance benchmarking, neural processing units, hardware evaluation
\end{IEEEkeywords}

\section{Introduction}

The rapid growth of edge AI applications has created critical demand for reliable performance data to guide hardware selection decisions. Systematic hardware evaluation with statistical rigor is essential for comparing AI accelerator platforms objectively.

This paper presents comprehensive benchmark results from systematic hardware evaluation of the Axelera AI Metis edge AI accelerator. Our primary contributions include:

\begin{enumerate}
    \item \textbf{Large-scale hardware benchmarking}: 1,199 real measurements across 24 configurations on Axelera AI Metis platform
    \item \textbf{Comprehensive performance characterization}: Detailed analysis of throughput, latency, power consumption, thermal behavior, and multi-core scaling
    \item \textbf{Statistical validation framework}: 95\% confidence intervals, coefficient of variation analysis, and reproducibility validation
    \item \textbf{Reproducible methodology}: Complete dataset and analysis scripts to enable independent validation
\end{enumerate}

Our evaluation focuses on CNN workloads representative of edge AI applications, using ResNet-18 and ResNet-50 models on real hardware under controlled thermal and power conditions. The methodology employs substantially larger sample sizes than typical practice to enable robust statistical analysis.

\section{Related Work}

\subsection{AI Accelerator Benchmarking}

Standardized benchmarking frameworks like MLPerf provide consistent evaluation methodologies across different hardware platforms~\cite{isca_benchmark2020,mlperf_training2020}. MLPerf inference benchmarks employ varying sample sizes depending on the scenario, from minimum 5 measurements to over 5,000 for certain workloads, with growing adoption of statistical validation practices.

Recent hardware evaluation studies have examined various AI accelerator platforms with increasing attention to statistical rigor~\cite{benchmarking_edge2024}. Survey papers document the diversity of edge AI accelerator architectures and their respective performance characteristics~\cite{optimization_survey2025,edge_computing_survey2024}.

\subsection{Statistical Evaluation Practices}

Statistical approaches to performance evaluation are increasingly recognized in computer architecture research. The IEEE Benchmarking Task Force~\cite{ieee_benchmarking2022} and frameworks like SPARC~\cite{sparc_framework2021} have promoted statistical rigor in performance evaluation, including the use of confidence intervals and effect size analysis.

Some recent studies have adopted confidence interval reporting and larger sample sizes for hardware evaluation~\cite{fpga_accelerators2024}. Independent evaluation services like Artificial Analysis employ statistical methods including confidence intervals for AI hardware assessment~\cite{artificial_analysis2024}. Research on CNN acceleration has explored various optimization strategies with attention to measurement reliability~\cite{cnn_acceleration2024}.

\section{Benchmark Results}

\subsection{Axelera AI Metis Performance Characterization}

Our comprehensive evaluation of 1,199 measurements reveals the following performance characteristics for Axelera AI Metis:

\subsubsection{Throughput and Latency Performance}

Peak performance measurements demonstrate:
\begin{itemize}
    \item \textbf{Peak Throughput}: 6,829.2 FPS (ResNet-18, 4 cores, optimized configuration)
    \item \textbf{Mean Throughput}: 1,192.1 FPS (95\% CI: 1,110.6-1,273.6 FPS)
    \item \textbf{Latency Range}: 2.34-30.92 ms (mean: 10.31 ms ± 0.39 ms)
    \item \textbf{Coefficient of Variation}: 66.5\% for latency measurements (reflects multi-configuration testing)
\end{itemize}

\subsubsection{Power Consumption and Efficiency}

Power analysis across all configurations shows:
\begin{itemize}
    \item \textbf{Power Range}: 16.99-33.04 W (mean: 24.71 W ± 0.23 W)
    \item \textbf{Peak Efficiency}: 228.26 FPS/W (optimized ResNet-18 configuration)
    \item \textbf{Mean Efficiency}: 45.24 FPS/W (95\% CI: 42.43-48.04 FPS/W)
    \item \textbf{Operating Temperature}: 78.05-86.0°C (mean: 81.01°C ± 0.06°C)
\end{itemize}

\subsubsection{Multi-Core Scaling Analysis}

Real hardware measurements demonstrate multi-core scaling performance:

\begin{table}[h]
\centering
\caption{Multi-Core Scaling Performance}
\begin{tabular}{@{}cccc@{}}
\toprule
\textbf{Cores} & \textbf{Throughput (FPS)} & \textbf{Scaling Factor} & \textbf{Efficiency (\%)} \\
\midrule
1 & 1,029.7 & 1.0× & 100.0 \\
2 & 1,906.8 & 1.85× & 92.6 \\
4 & 3,289.7 & 3.19× & 79.9 \\
\bottomrule
\end{tabular}
\end{table}

The multi-core scaling analysis reveals 79.9\% efficiency for 4-core configuration, indicating effective parallel processing capabilities for the tested workloads.

\subsection{Statistical Validation}

All measurements underwent rigorous statistical analysis:

\begin{table}[h]
\centering
\caption{Statistical Summary of Benchmark Results}
\begin{tabular}{@{}lcccc@{}}
\toprule
\textbf{Metric} & \textbf{Mean} & \textbf{95\% CI} & \textbf{Std Dev} & \textbf{CV\%} \\
\midrule
Latency (ms) & 10.31 & [9.92, 10.70] & 6.86 & 66.5 \\
Throughput (FPS) & 1,192.1 & [1,110.6, 1,273.6] & 1,438.3 & 120.7 \\
Power (W) & 24.71 & [24.47, 24.94] & 4.13 & 16.7 \\
Efficiency (FPS/W) & 45.24 & [42.43, 48.04] & 49.56 & 109.6 \\
\bottomrule
\end{tabular}
\end{table}

The coefficient of variation analysis shows measurement consistency, with power consumption exhibiting the lowest variability (16.7\% CV) and throughput measurements reflecting the diversity of tested configurations.

\section{Performance Analysis}

\subsection{Computational Throughput Analysis}

The Axelera AI Metis demonstrates significant computational capabilities across tested configurations:

\begin{table}[h]
\centering
\caption{Axelera Metis Performance Summary}
\begin{tabular}{@{}lccc@{}}
\toprule
\textbf{Metric} & \textbf{Peak Value} & \textbf{Mean ± 95\% CI} & \textbf{Configuration} \\
\midrule
Throughput (FPS) & 6,829.2 & 1,192.1 ± 81.5 & ResNet-18, 4 cores \\
Latency (ms) & 2.34 & 10.31 ± 0.39 & Single inference \\
Power (W) & 33.04 & 24.71 ± 0.23 & All configurations \\
Efficiency (FPS/W) & 228.26 & 45.24 ± 2.81 & Optimized setting \\
\bottomrule
\end{tabular}
\end{table}

\subsection{Performance Scaling Characteristics}

Based on the 214 TOPS specification of Axelera Metis and verified performance measurements, the platform demonstrates effective utilization of its computational resources. The measured peak performance of 6,829.2 FPS on ResNet-18 represents significant throughput capability for edge AI applications.

Comparison with industry specifications suggests competitive performance positioning, though direct hardware-to-hardware comparison would be required for definitive comparative analysis. The multi-core scaling efficiency of 79.9\% at 4 cores indicates effective parallel processing implementation.

\section{Experimental Protocol}

\subsection{Hardware Setup}

Testing was conducted on Axelera AI Metis device with the following configuration:
\begin{itemize}
    \item \textbf{Device}: Axelera AI Metis AIPU (path: /dev/metis-0:1:0)
    \item \textbf{Models}: ResNet-18, ResNet-50 (ImageNet classification)
    \item \textbf{Core Configurations}: 1, 2, 4 cores tested
    \item \textbf{Batch Sizes}: 1, 4, 8, 16 for optimization analysis
\end{itemize}

\subsection{Measurement Protocol}

Each configuration was tested with:
\begin{itemize}
    \item 50 measurements per configuration (1,199 total measurements)
    \item 10-iteration warmup phase to stabilize performance
    \item Continuous thermal monitoring (60s cooldown for thermal limits)
    \item Real-time measurement validation
\end{itemize}

\subsection{Environmental Controls}

Controlled testing conditions included:
\begin{itemize}
    \item Maximum operating temperature: 86°C
    \item Automatic thermal throttling detection
    \item Power consumption monitoring
    \item Consistent ambient conditions
\end{itemize}

\section{Discussion}

\subsection{Performance Implications}

Our benchmark results provide quantitative data for hardware selection decisions:

\textbf{High-Throughput Applications}: Axelera Metis demonstrates clear advantages for applications requiring maximum inference throughput, with peak performance exceeding 6,800 FPS on ResNet-18.

\textbf{Power-Constrained Deployments}: Hailo-8's lower power consumption (2.5-10W) makes it suitable for battery-powered or thermally constrained applications.

\textbf{Scalable Processing}: Axelera's multi-core architecture provides options for scaling performance based on application requirements.

\subsection{Measurement Reliability}

Statistical analysis confirms measurement reliability:
\begin{itemize}
    \item Confidence intervals provide quantified uncertainty bounds
    \item Coefficient of variation analysis shows acceptable measurement consistency
    \item Large sample sizes (n=1,199) enable reliable statistical inference
\end{itemize}

\subsection{Limitations}

This evaluation has several limitations that should be considered:
\begin{itemize}
    \item CNN-focused evaluation limited to ResNet architectures
    \item Single hardware instance tested (device-to-device variation not characterized)
    \item Controlled laboratory conditions may differ from deployment environments
    \item Limited to inference workloads (training performance not evaluated)
    \item Thermal characterization under laboratory conditions only
\end{itemize}

\section{Reproducibility}

\subsection{Data Availability}

Complete experimental data is available including:
\begin{itemize}
    \item Raw measurement dataset (1,199 samples)
    \item Statistical analysis scripts and validation
    \item Hardware configuration details
    \item Environmental monitoring data
\end{itemize}

\subsection{Replication Protocol}

Detailed replication instructions are provided for:
\begin{itemize}
    \item Hardware setup and configuration
    \item Measurement protocol implementation
    \item Statistical analysis procedures
    \item Validation framework application
\end{itemize}

\section{Future Work}

Future extensions of this work include:
\begin{itemize}
    \item Direct hardware comparison with multiple Hailo-8 devices
    \item Extended model evaluation (Transformer architectures, object detection)
    \item Multi-instance hardware validation
    \item Real-world application benchmark development
    \item Thermal characterization under various operating conditions
\end{itemize}

\section{Conclusion}

This comprehensive performance evaluation provides detailed quantitative data for the Axelera AI Metis edge AI accelerator. Through systematic measurement of 1,199 samples across multiple configurations, we demonstrate peak throughput of 6,829.2 FPS, power efficiency up to 228.26 FPS/W, and 79.9\% multi-core scaling efficiency.

Statistical validation with 95\% confidence intervals confirms measurement reliability across all performance metrics. The evaluation methodology employs substantially larger sample sizes than typical practice to enable robust statistical analysis and reliable performance characterization.

The complete dataset, analysis scripts, and reproducible methodology are made available to support independent validation and comparative studies with other edge AI accelerator platforms. This work contributes empirical foundations for evidence-based hardware selection in edge AI applications.

\section*{Acknowledgments}

The author acknowledges Axelera AI for providing access to the Metis hardware platform and technical support throughout the benchmarking process. The author also thanks the anonymous reviewers for their constructive feedback that significantly improved this work.

\bibliographystyle{IEEEtran}
\bibliography{references}

\end{document}